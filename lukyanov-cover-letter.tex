%%%%%%%%%%%%%%%%%%%%%%%%%%%%%%%%%%%%%%%%%
% "ModernCV" CV and Cover Letter
% LaTeX Template
% Version 1.1 (9/12/12)
%
% This template has been downloaded from:
% http://www.LaTeXTemplates.com
%
% Original author:
% Xavier Danaux (xdanaux@gmail.com)
%
% License:
% CC BY-NC-SA 3.0 (http://creativecommons.org/licenses/by-nc-sa/3.0/)
%
% Important note:
% This template requires the moderncv.cls and .sty files to be in the same 
% directory as this .tex file. These files provide the resume style and themes 
% used for structuring the document.
%
%%%%%%%%%%%%%%%%%%%%%%%%%%%%%%%%%%%%%%%%%

%----------------------------------------------------------------------------------------
% PACKAGES AND OTHER DOCUMENT CONFIGURATIONS
%----------------------------------------------------------------------------------------

\documentclass[11pt,a4paper,roman]{moderncv} % Font sizes: 10, 11, or 12; paper sizes: a4paper, letterpaper, a5paper, legalpaper, executivepaper or landscape; font families: sans or roman

\moderncvstyle{casual} % CV theme - options include: 'casual' (default), 'classic', 'oldstyle' and 'banking'
\moderncvcolor{blue} % CV color - options include: 'blue' (default), 'orange', 'green', 'red', 'purple', 'grey' and 'black'

\usepackage{ragged2e}

\usepackage[scale=0.8]{geometry} % Reduce document margins
%\setlength{\hintscolumnwidth}{3cm} % Uncomment to change the width of the dates column
%\setlength{\makecvtitlenamewidth}{10cm} % For the 'classic' style, uncomment to adjust the width of the space allocated to your name

%----------------------------------------------------------------------------------------
% NAME AND CONTACT INFORMATION SECTION
%----------------------------------------------------------------------------------------

\firstname{Georgy} % Your first name
\familyname{Lukyanov} % Your last name

% All information in this block is optional, comment out any lines you don't need
  % \address{44 Pushkinskaya st.}{344082 Rostov-on-Don, Russia}
% \mobile{+7(908) 506 9512}
% \email{georgiylukjanov@gmail.com}
% \homepage{geo2a.github.io}{geo2a.github.io}

\nopagenumbers 

\begin{document}

%----------------------------------------------------------------------------------------
% COVER LETTER
%----------------------------------------------------------------------------------------

% To remove the cover letter, comment out this entire block

\clearpage

\recipient{\ }{\ } % Letter recipient
\date{\today} % Letter date
\opening{Dear Sir or Madam,} % Opening greeting
\closing{Yours faithfully,} % Closing phrase
% \enclosure[Attached]{curriculum vit\ae{}} % List of enclosed documents

\makelettertitle % Print letter title
\pagestyle{empty}

% \vspace{-2em}

\justify
I've been fascinated by functional programming since my third year of
undergraduate studies, because it takes roots in highly abstract fields of 
mathematics such as logic and theory of computation. Being attracted 
by its abstract beauty and solid foundation,
I've started to look for ways to improve my understanding of mathematics that 
makes a basis for it and to have a supervised research work on some connected 
topic. And, fortunately, there were faculty members in my department who shared 
my interests and helped me to develop myself and my knowledge.

What made me so found of functional programming languages is its expressiveness
and high degree of control under program behaviour. And more than that,
the most notable distinction of modern functional programming languages is the
ability to check correctness of programs even without the need to run them,
enabling software developers with possibility to eliminate a lot of errors and
save substantial amount of time, because correction of programming errors
(debugging) requires a lot more resources then initial development.

So, I've been doing research on functional programming under the supervision of
Prof Artem Pelenitsyn. And, after two years, this activities led 
my Bachelor's thesis in Applied Mathematics and Informatics, that discussed
different approaches to text parsing using functional programming and 
computational effects. 
With this project, I won the first place award in student research competition
"Week of Science 2015" of Southern Federal University (my Alma Mater) and
received excellent mark (5 of 5) for my Bachelor's thesis defence. 
Recently, I've revised this work, translated into English and submitted to 
TMPA 2017 conference (\url{http://tmpaconf.org/}) under the title "Functional 
parser of Markdown language based on monad combining and monoidal source stream 
representation" and it has been accepted. I'll be presenting it on 3-4
March, 2017 in Moscow, proceedings are going to be published by Springer in
their Communications in Computer and Information Science series (indexed by 
Scopus and DBLP digital libraries). 

After receiving my Bachelor's diploma, I've enrolled to Master's
programme "Fundamental Informatics and Computer Science" in the same university 
(Southern Federal University, Russia) and have been accepted without
examination, thanks to my high research activity during undergraduate studies
and high GPA (4.53 / 5.0). I've continued to work with my supervisor, 
Prof Artem Pelenitsyn. During the next year, I've been developing
"Student's Big Brother" --- a distributed system, that helps teacher of 
programming classes to supervise large practical sessions 
(Project description is available on github:
\url{https://github.com/geo2a/students-big-brother}). This project was motivated
by quite big sizes of freshmen groups and a fact that most first year students 
are too shy to ask questions if they are stuck. 

Concurrently with working on the above-mentioned project, I've been continuing
to study the topic of my Bachelor's thesis: computational effects. My current
activities include development of parsers library based on experimental
programming language Frank, prototype is available on GitHub: 
\url{https://github.com/geo2a/frankoparsec}. I'm going to submit a paper to
Programming Languages and Compilers (http://plc.sfedu.ru/) conference, 
which is going to be held at my university in April, 2017.  

Alongside with development of these projects and following my own classes, 
I've been assisting to teach "Functional Programming in Haskell" course taught 
to undergrads in Southern Federal University by Prof Vitaly Bragilevskiy. 
I've been teaching practical classes with occasional lecturing.
It wasn't my first experience of teaching, because I've already been
giving some lessons on mathematics and programming individually to freshmen 
and school pupils, but it is a first one of such a large scale. 
It was very engaging to give lectures on topics of my interest and
help students to dive into world of functional programming during practical 
classes. Besides, I enjoyed working with highly motivated students, 
especially when it came to personal project section of the course:
students worked in pairs developing video game of their choice. 
As I think, it was really important for me to have a look on educational 
procedures from teachers point of view. Furthermore, this experience was main
factor that motivated me to consider a career in academia, and, as a natural step
towards, to find myself a solid PhD programme.

For second semester of my Master studies, I've applied and have been accepted 
for an academic exchange programme (Erasmus+) at Vilnius University, Lithuania. 
I've been following courses in English, meeting faculty members and engineers
from local software companies. This experience opened my mind to the world, and,
that is of huge value, made me confident about my language and communication
skills. This exchange programme was a second factor, that made me confident that
I'm able to undertake such a challenge as PhD in Europe.

After I've come back home, I've started to search for a suitable PhD-position.
An ideal position I have been looking for, would include research activities
and practical software development connected with application of typed 
functional programming to some challenging engineering problems, that require
high level of reliability.

I've found Dr Mokhov's call for PhD students and got very excited about the 
proposed topic. He proposed me to take part in project NSS1742 "Formal Methods
for Design of Safety-Critical Processors". That was exactly what I was looking
for, because it incorporates both study of formal methods and practice in
development of large-scale systems. Since my main research interest is
functional programming languages and I have a solid software development
background, Dr Mokhov proposed to focus on domain-specific languages for
processor specification with formal semantics and hardware
compilation/generation tools. I immediately joined the on-going research work
by Dr Mokhov and his research team on the topic, which led to our joint paper
"Prototyping Resilient Processing Cores in Workcraft" which was accepted at the
2nd International Workshop on Resiliency in Embedded Electronic Systems that is
a part of DATE'17 --- the top European conference on Electronic Design
Automation that will take place in Lausanne, Switzerland, in March 2017.
We are currently preparing an extended version of the paper, which will be
published by IEEE, and my section on domain-specific languages plays the
central role in the paper. Considering all mentioned facts, our interaction has
been highly productive and I would be glad to work under Dr Mokhov's
supervision.

I feel fascinated by the perspective of being part of a productive academic
team and taking part in research on processor architecture development tools 
using cutting-edge programming techniques. In future, having such an experience, 
academic connections and degree, that PhD studies in Newcastle University 
provide, it'll be possible for me to apply for research positions in industry or
even pursue an academic job. It is yet to be decided.

\makeletterclosing % Print letter signature

%----------------------------------------------------------------------------------------

\end{document}