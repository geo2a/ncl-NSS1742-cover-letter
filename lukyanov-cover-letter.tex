%%%%%%%%%%%%%%%%%%%%%%%%%%%%%%%%%%%%%%%%%
% "ModernCV" CV and Cover Letter
% LaTeX Template
% Version 1.1 (9/12/12)
%
% This template has been downloaded from:
% http://www.LaTeXTemplates.com
%
% Original author:
% Xavier Danaux (xdanaux@gmail.com)
%
% License:
% CC BY-NC-SA 3.0 (http://creativecommons.org/licenses/by-nc-sa/3.0/)
%
% Important note:
% This template requires the moderncv.cls and .sty files to be in the same 
% directory as this .tex file. These files provide the resume style and themes 
% used for structuring the document.
%
%%%%%%%%%%%%%%%%%%%%%%%%%%%%%%%%%%%%%%%%%

%----------------------------------------------------------------------------------------
% PACKAGES AND OTHER DOCUMENT CONFIGURATIONS
%----------------------------------------------------------------------------------------

\documentclass[11pt,a4paper,sans]{moderncv} % Font sizes: 10, 11, or 12; paper sizes: a4paper, letterpaper, a5paper, legalpaper, executivepaper or landscape; font families: sans or roman

\moderncvstyle{casual} % CV theme - options include: 'casual' (default), 'classic', 'oldstyle' and 'banking'
\moderncvcolor{blue} % CV color - options include: 'blue' (default), 'orange', 'green', 'red', 'purple', 'grey' and 'black'

\usepackage{lipsum} % Used for inserting dummy 'Lorem ipsum' text into the template

\usepackage[scale=0.75]{geometry} % Reduce document margins
%\setlength{\hintscolumnwidth}{3cm} % Uncomment to change the width of the dates column
%\setlength{\makecvtitlenamewidth}{10cm} % For the 'classic' style, uncomment to adjust the width of the space allocated to your name

%----------------------------------------------------------------------------------------
% NAME AND CONTACT INFORMATION SECTION
%----------------------------------------------------------------------------------------

\firstname{Georgy} % Your first name
\familyname{Lukyanov} % Your last name

% All information in this block is optional, comment out any lines you don't need
\title{Curriculum Vitae}
\address{44 Pushkinskaya st.}{344082 Rostov-on-Don, Russia}
\mobile{+7(908) 506 9512}
\email{georgiylukjanov@gmail.com}
\homepage{geo2a.github.io}{geo2a.github.io}

\begin{document}

%----------------------------------------------------------------------------------------
% COVER LETTER
%----------------------------------------------------------------------------------------

% To remove the cover letter, comment out this entire block

\clearpage

\recipient{HR Departmnet}{Corporation\\123 Pleasant Lane\\12345 City, State} % Letter recipient
\date{\today} % Letter date
\opening{Dear Sir or Madam,} % Opening greeting
\closing{Yours faithfully,} % Closing phrase
% \enclosure[Attached]{curriculum vit\ae{}} % List of enclosed documents

\makelettertitle % Print letter title

A distinctive feature of functional programming paradigm is that it takes
roots in such highly abstract fields of mathematics as logic and theory of
computations. Influenced with concision of lambda calculus and clarity of logic, 
functional programming languages have laconic syntax assigned to powerful
and rich semantics. And more than that, the most notable distinction of modern 
functional programming languages are its type systems, providing feature of
so-called lightweight verification and enabling software developers with
possibility to eliminate a lot of errors without even running their programs.

Those facts about functional programming have fascinated me when I was in third
year of my undergraduate studies and I've started to look for a way to improve 
my understanding of mathematics that makes a basis for it and to have a
supervised research work on some topic, connected with functional programming. 
And, fortunately, there faculty members in my department who shared my interests
and helped me to develop myself and my knowledge. Finally, after two years, 
it led to creation of my Bachelor's thesis, that discussed different approaches
to typing of computations containing side-effects to development of parsers.
Recently, I've revised this work, translated into English and submitted to 
TMPA 2017 conference (\url{http://tmpaconf.org/}) and it was accepted. I'll be
presenting it on 3d or 4th of March, 2017 in Moscow, proceedings are going to
be published with Springer in their Communications in Computer and Information 
Science series (indexed by Scopus and DBLP digital libraries). 

After receiving my Bachelor's diploma, I've enrolled to Master's
program, provided by the same university and continued to work with my
supervisor. During next year, I've been developing a distributed system, 
that helps teacher of programming classes to distribute his attention between
students (Project description is available on github: \url{https://github.com/geo2a/students-big-brother}). 

Alongside with development of this project and following my own classes, 
I've been assisting to teach "Functional Programming in Haskell" course taught 
by prof. Vitaly Bragilevskiy to undergrads by occasional lecturing
and teaching practical classes. It wasn't my first experience of teaching,
because I've already been given some lessons on mathematics and programming 
individually to freshmen and school pupils, but it first one of such a large 
scale. As I think, it was really important for me to have a look on educational 
procedures from teachers point of view.

Then, I was advised to go for an academic exchange programme (Erasmus+) to 
Vilnius University in Lithuania. I've applied and have been accepted, so I've
spent second semester of my Master studies in Vilnius university, following
courses in English, meeting faculty members and engineers from local software
companies. This experience opened my mind to the world, and, that is of huge
value, made me confident about my language and communication skills. Actually, 
it was this exchange programme, that made me feel like I'm able to undertake 
such a challenge as an oversea PhD.

After I've come back home, I've started to search for a suitable PhD-position.
An ideal position I have been looking for, would include a research activities
and practical software development connected with application of typed 
functional programming to some challenging engineering problems, that require
high level of reliability. Besides, I was looking for a funded position, because
otherwise I wouldn't be able to cover tuition fees and living expenses.

As far as I'm highly interested in functional programming and also in software 
development with Haskell, I try to be connected to community. It happened so, 
that the most suitable way for me to follow for researchers updates are their 
Twitter accounts and some mailing lists (such as TYPES). Fortunately, once, 
one of the people who I've been following, Dr. Andrey Mokhov, 
published a request, stating that he is looking for PhD-students to work with 
him in Newcastle University. I wrote him an email, 
supplying my CV, and we started a discussion. He proposed me to take part in 
a project NSS1742 "Formal Methods for Design of Safety-Critical Processors". 
That was exactly what I was looking for, because it incorporates both study of 
formal methods and practice in development of large-scale systems. As far as
my main research interest is functional programming languages and I have a solid
software development background, Dr. Mokhov proposed to work on domain-specific
languages for processor specification with formal semantics and hardware
compilation/generation tools. Dr. Mokhov even  gave me a possibility to 
contribute into work of his research group and write a section names 
"DSLs for improved design productivity" of paper "Prototyping Resilient 
Processing Cores in Workcraft" that has been accepted for 2nd International
Workshop on Resiliency in Embedded Electronic Systems 2017 
(\url{https://www.edacentrum.de/rees}). Considering all mentioned facts, our
interactions seems highly productive to me and I would be glad to work under 
his supervision. Moreover, Newcastle university has great scholarship 
possibilities, therefore I hope to be able to have enough means to feel
comfortable. 

I feel fascinated by perspective of being part of productive academic team and
taking part in research on processor architecture development tools 
using cutting-edge programming techniques. In future, having such an experience, 
academic connections and degree, that PhD studies in Newcastle University may 
provide, it'll be possible for me to apply for research positions in industry or
even pursue an academic job. It is yet to be decided.

\makeletterclosing % Print letter signature

%----------------------------------------------------------------------------------------

\end{document}