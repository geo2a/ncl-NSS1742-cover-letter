%%%%%%%%%%%%%%%%%%%%%%%%%%%%%%%%%%%%%%%%%
% "ModernCV" CV and Cover Letter
% LaTeX Template
% Version 1.1 (9/12/12)
%
% This template has been downloaded from:
% http://www.LaTeXTemplates.com
%
% Original author:
% Xavier Danaux (xdanaux@gmail.com)
%
% License:
% CC BY-NC-SA 3.0 (http://creativecommons.org/licenses/by-nc-sa/3.0/)
%
% Important note:
% This template requires the moderncv.cls and .sty files to be in the same 
% directory as this .tex file. These files provide the resume style and themes 
% used for structuring the document.
%
%%%%%%%%%%%%%%%%%%%%%%%%%%%%%%%%%%%%%%%%%

%----------------------------------------------------------------------------------------
% PACKAGES AND OTHER DOCUMENT CONFIGURATIONS
%----------------------------------------------------------------------------------------

\documentclass[11pt,a4paper,sans]{moderncv} % Font sizes: 10, 11, or 12; paper sizes: a4paper, letterpaper, a5paper, legalpaper, executivepaper or landscape; font families: sans or roman

\moderncvstyle{casual} % CV theme - options include: 'casual' (default), 'classic', 'oldstyle' and 'banking'
\moderncvcolor{blue} % CV color - options include: 'blue' (default), 'orange', 'green', 'red', 'purple', 'grey' and 'black'

\usepackage{ragged2e}

\usepackage[scale=0.75]{geometry} % Reduce document margins
%\setlength{\hintscolumnwidth}{3cm} % Uncomment to change the width of the dates column
%\setlength{\makecvtitlenamewidth}{10cm} % For the 'classic' style, uncomment to adjust the width of the space allocated to your name

%----------------------------------------------------------------------------------------
% NAME AND CONTACT INFORMATION SECTION
%----------------------------------------------------------------------------------------

\firstname{Georgy} % Your first name
\familyname{Lukyanov} % Your last name

% All information in this block is optional, comment out any lines you don't need
\title{Curriculum Vitae}
% \address{44 Pushkinskaya st.}{344082 Rostov-on-Don, Russia}
% \mobile{+7(908) 506 9512}
% \email{georgiylukjanov@gmail.com}
% \homepage{geo2a.github.io}{geo2a.github.io}

\begin{document}

%----------------------------------------------------------------------------------------
% COVER LETTER
%----------------------------------------------------------------------------------------

% To remove the cover letter, comment out this entire block

\clearpage

\recipient{HR Departmnet}{Corporation\\123 Pleasant Lane\\12345 City, State} % Letter recipient
\date{\today} % Letter date
\opening{Dear Sir or Madam,} % Opening greeting
\closing{Yours faithfully,} % Closing phrase
% \enclosure[Attached]{curriculum vit\ae{}} % List of enclosed documents

\makelettertitle % Print letter title

\justify
I've been fascinated by functional programming since my third year of
undergraduate studies, because it takes roots in such highly
abstract fields of mathematics as logic and theory of computations. Being hooked 
by its abstract beauty and solid foundation,
I've started to look for a way to improve my understanding of mathematics that 
makes a basis for it and to have a supervised research work on some connected 
topic. And, fortunately, there were faculty members in my department who shared 
my interests and helped me to develop myself and my knowledge.

What made my so found of functional programming languages is its expressiveness
and high degree of control under program behaviour. And more than that,
the most notable distinction of modern functional programming languages is
ability to check correctness of programs even without need to run them,
enabling software developers with possibility to eliminate a lot of errors and
save massive amounts of time, because correction of programming errors
(debugging) requires a lot more resources then initial development.

So, I've been doing research on functional programming under supervision of
prof Artem Pelenitsyn. And, after two years, this activities led to creation of
my Bachelor's thesis in Applied Mathematics and Informatics, that discussed
different approaches to typing of
computations containing side-effects in context of development of parsers. 
With this project, I won first place award in students research competition
"Week of Science 2015" of Southern Federal University (my Alma Mater) and
received excellent mark (5 of 5) for my Bachelor's thesis defence. 
Recently, I've revised this work, translated into English and submitted to 
TMPA 2017 conference (\url{http://tmpaconf.org/}) under the title "Functional 
parser of Markdown language based on monad combining and monoidal source stream 
representation" and it has been accepted. I'll be presenting it on 3d or 4th of
March, 2017 in Moscow, proceedings are going to be published by Springer in
their Communications in Computer and Information Science series (indexed by 
Scopus and DBLP digital libraries). 

After receiving my Bachelor's diploma, I've enrolled to Master's
programme "Fundamental Informatics and Computer Science" in the same university 
(Southern Federal University, Russia) and have been accepted without
examination, thanks to my high research activity during undergraduate studies
and high GPA (4.53 / 5.0). I've continued to work with my supervisor, 
prof Artem Pelenitsyn. During next year, I've been developing
"Student's Big Brother" --- a distributed system, that helps teacher of 
programming classes to distribute his attention between students 
(Project description is available on github:
\url{https://github.com/geo2a/students-big-brother}). This project was motivated
by quite big sizes of freshmen groups and a fact that most first year students 
are too shy to ask questions if they are stuck. So, teacher has to try to watch
student's activity during practical classes by looking into his display. This
procedure is quite time consuming. Therefore, proposed workflow is as follows:
students write code and special program grabs their sources and sends to the
server when source code files are stored in database for further analysis and
access. Then, source files got displayed in teachers web interface, grouped by
student. Updates are requested from a server by teacher's demand. System consist
of three modules: application server and student's module are implemented in
Haskell language using modern programming techniques, and teachers web interface
is implemented via standard web technologies (HTML/CSS/JavaScript). In addition,
there is also a prototype of iOS mobile app for teacher implemented in Swift.  

Concurrently with doing web programming in Haskell. I've been continuing to
study the topic of my Bachelor's thesis: computational effects. My current
activities include development of parsers library based on experimental
programming language Frank, prototype is available on GitHub: 
\url{https://github.com/geo2a/frankoparsec}. I'm going to submit a paper to
Programming Languages and Compilers (http://plc.sfedu.ru/) conference, 
which is going to be held at my university in April, 2017.  

Alongside with development of these projects and following my own classes, 
I've been assisting to teach "Functional Programming in Haskell" course taught 
to undergrads in Southern Federal University by prof. Vitaly Bragilevskiy. 
I've been teaching practical classes with occasional lecturing.
It wasn't my first experience of teaching, because I've already been
giving some lessons on mathematics and programming individually to freshmen 
and school pupils, but it is a first one of such a large scale. 
It was very engaging to give lectures on topics of my interest and
help students to dive into world of functional programming during practical 
classes. Besides, I enjoyed working with highly motivated students, 
especially when it came to personal project section of course:
students worked in pairs developing video game of their choice. 
As I think, it was really important for me to have a look on educational 
procedures from teachers point of view. Furthermore, this experience was main
factor that enabled me to consider a career in academia, and, as a natural step
towards, to find myself a solid PhD programme.

For second semester of my Master studies, I've applied and have been accepted 
for an academic exchange programme (Erasmus+) to Vilnius University, Lithuania. 
I've spent  in Vilnius university, following courses in English, meeting 
faculty members and engineers from local software companies. This experience 
opened my mind to the world, and, that is of huge value, made me confident 
about my language and communication skills. This exchange programme was a second
factor, that made me feel like I'm able to undertake such a challenge as an 
oversea PhD.

After I've come back home, I've started to search for a suitable PhD-position.
An ideal position I have been looking for, would include research activities
and practical software development connected with application of typed 
functional programming to some challenging engineering problems, that require
high level of reliability. Besides, I was looking for a funded position, because
otherwise I wouldn't be able to cover tuition fees and living expenses.

I've found Dr Mokhov's call for PhD students and got very excited about the 
proposed topic. I wrote him an email, supplying my CV, and we started a 
discussion. He proposed me to take part in project NSS1742 "Formal Methods for 
Design of Safety-Critical Processors". That was exactly what I was looking for, 
because it incorporates both study of formal methods and practice in development
of large-scale systems. As far as my main research interest is functional
programming languages and I have a solid software development background,
Dr Mokhov proposed to work on domain-specific languages for processor
specification with formal semantics and hardware compilation/generation tools.
I immediately joined the on-going research work by Dr Mokhov and his research
team on the topic, which led to our joint paper "Prototyping Resilient 
Processing Cores in Workcraft" which was accepted at the 2nd International
Workshop on Resiliency in Embedded Electronic Systems that is a part of DATE'17
--- the top European conference on Design Automation that will take place in
Lausanne, Switzerland, in March 2017. We are currently preparing an extended
version of the paper, which will be published by IEEE, and my section
on domain-specific languages plays the central role in the paper.

Considering all mentioned facts, our interaction seems highly productive to me
and I would be glad to work under Dr Mokhov's supervision.

I feel fascinated by perspective of being part of productive academic team and
taking part in research on processor architecture development tools 
using cutting-edge programming techniques. In future, having such an experience, 
academic connections and degree, that PhD studies in Newcastle University 
provide, it'll be possible for me to apply for research positions in industry or
even pursue an academic job. It is yet to be decided.

\makeletterclosing % Print letter signature

%----------------------------------------------------------------------------------------

\end{document}